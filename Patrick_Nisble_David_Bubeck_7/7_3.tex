\documentclass[12pt, a4paper, reqno]{amsart}
\usepackage{amsmath}
\usepackage{a4wide}
\usepackage[ngerman]{babel}
\usepackage[utf8]{inputenc}

\begin{document}
    \title{7.3: algorithmische Komplexität}
    \maketitle
    
    \section*{a)}
    für n=32 ist t=5s, was ist t für n=64 unter bestimmten Komplexitäten
    
    \subsection*{i)} $log_2(n)$
    
    \begin{align}
        &k\cdot log_2(32) = 5\\
        \Leftrightarrow & k = 1 \\
        \Rightarrow & log_2(64) = 6
    \end{align}
    Der Algorithmus benötigt 6s für 64 Elemente
    
    \subsection*{ii)} $n$
    
    \begin{align}
    &k\cdot 32 = 5\\
    \Leftrightarrow & k = \frac 5 {32} \\
    \Rightarrow & \frac 5 {32}\cdot 64 = 10
    \end{align}
    Der Algorithmus benötigt 10s für 64 Elemente
    
    
    \subsection*{iii)} $n\cdot log_2(n)$
    
    \begin{align}
    &k\cdot 32 \cdot log_2(32) = 5\\
    \Leftrightarrow & k = \frac 1 {32} \\
    \Rightarrow & 64\cdot log_2(64) = 12
    \end{align}
    Der Algorithmus benötigt 12s für 64 Elemente
    
    
    \subsection*{iv)} $n^2$
    
    \begin{align}
    &k\cdot 32^2 = 5\\
    \Leftrightarrow & k = \frac 5 {964} \\
    \Rightarrow & \frac 5 {964} \cdot 64^2 = 20
    \end{align}
    Der Algorithmus benötigt 20s für 64 Elemente
    
    \subsection*{v)} $2^n$
    
    \begin{align}
    &k\cdot 2^{32} = 5\\
    \Leftrightarrow & k = \frac 5 {2^{32}} \\
    \Rightarrow & \frac 5 {2^{32}} \cdot 2^{64} = 5\cdot 2^{32}
    \end{align}
    Der Algorithmus benötigt ca. 640 Jahre für 64 Elemente
    
    
    \section*{b)}
    z.z: $\forall a,b>0: log_a(n) \in O(log_b(n)) = \{ f(n)\ |\ f(n)<k\cdot log_b(n), k\in \mathbb R, \text{ ab einem Punkt } n_0 \}$
    
    \begin{align}
        \xrightarrow[nach Log.Ges.]{} log_a(n)&=\frac{log_b(n)}{log_b(a)}\\
        \text{ mit } log_b(a) &= const.\\
        \Rightarrow \frac{log_b(n)}{log_b(a)} &= k\cdot log_b(n),\\
        \text{ mit } k&=\frac 1 {log_b(a)}\\
        \Rightarrow log_a(n)&\in O(log_b(n))
    \end{align}
    
    \section*{c)}
    Annahme: (effizient bis ineffizient) $log(n) < n^{1/2} < n\cdot log(n) < n^2 < 2^n$
    
    Test:
    \begin{itemize}
        \item $log(n), n^{1/2}$
        \begin{align}
            \lim_{n\to \infty}&\frac{log(n)}{n^{1/2}} = 0\\
            \Rightarrow &log(n) < n^{1/2} 
        \end{align}
        
        \item $n^{1/2}, n\cdot log(n)$
        \begin{align}
        \lim_{n\to \infty}&\frac{n^{1/2}}{n\cdot log(n)} = 0\\
        \Rightarrow &n^{1/2}<n\cdot log(n) 
        \end{align}
        
        \item $n\cdot log(n), n^2$
        \begin{align}
        \lim_{n\to \infty}&\frac{n\cdot log(n)}{n^{2}} = 0\\
        \Rightarrow &n\cdot log(n) < n^2 
        \end{align}
        
        \item $n^2, 2^n$
        \begin{align}
        \lim_{n\to \infty}&\frac{n^{2}}{2^n} = \lim_{n\to \infty}\frac{2n}{2^n\cdot ln(n)} = \lim_{n\to \infty}\frac{2}{2^n(ln^2(n)+\frac 1 n)}= 0\\
        \text{da } \lim_{n\to \infty} &\left( ln^2(n)+\frac 1 n \right)=\infty\\
        \Rightarrow &n\cdot n^2 < 2^n 
        \end{align}
    \end{itemize}
    $\Rightarrow$ die Annahme war Richtig (mit transitien Eigenschaften auf $\mathbb N$)
\end{document}